\documentclass{resume} % Use the custom resume.cls style
\usepackage{mathpazo}
\usepackage[left=0.6in,top=0.6in,right=0.6in,bottom=0.6in]{geometry} % Document margins
\newcommand{\tab}[1]{\hspace{.2667\textwidth}\rlap{#1}}
\newcommand{\itab}[1]{\hspace{0em}\rlap{#1}}
\name{Raul Guarini Riva} % Your name
\address{6401 N Sheridan Rd \#507, Chicago - IL 60626} % Your address
%\address{123 Pleasant Lane \\ City, State 12345} % Your secondary addess (optional)
\address{+1 917 535 5879 \\ raul.riva@kellogg.northwestern.edu} % Your phone number 
%\address{Nationality: Brazilian \\ Immigration Status: F-1 type Visa}
%\address{Nationality: Brazilian \\ Immigration Status: F1-type Student}
\usepackage[utf8]{inputenc}
\usepackage[english]{babel}

\begin{document}

%----------------------------------------------------------------------------------------
%	EDUCATION SECTION
%----------------------------------------------------------------------------------------

\begin{rSection}{Education}
{\bf Northwestern University} \hfill {\em Sep 2019 - March 2025 (expected)}
\\ PhD Candidate in Finance - currently in the 5th Year\\
Advisors: Torben Andersen and Viktor Todorov

{\bf Funda\c{c}\~ao Getulio Vargas (FGV-EPGE / Brazil)} \hfill {\em January 2017 - March 2019}
\\ M.Sc. in Economics \\ Advisors: Caio Almeida and Yuri Saporito

{\bf Fundação Getulio Vargas (FGV-EPGE / Brazil)} \hfill {\em January 2013 - December 2016}
\\ B.A. in Economics (\textit{summa cum laude}) \\ Advisor: Cecilia Machado
\end{rSection}

% Publications
\begin{rSection}{Publications}
\begin{rSubsection}{Intraday Cross-Sectional Distributions of Systematic Risk}{}{(with Torben Andersen, Martin Thyrsgaard and Viktor Todorov,  at the \underline{Journal of Econometrics})}{}

	Best paper award at the 2021 SoFiE Conference in Cambridge, UK.
	
	We develop a test for the detection of intraday changes in the cross-sectional distribution of assets’ exposure to observable factors. The test is constructed for a panel of high-frequency asset returns, with the size of the cross-section and the sampling frequency increasing simultaneously. It is based on a comparison of the empirical characteristic functions of estimates of the assets’ factor loadings at different parts of the trading day, formed from local blocks of asset returns and the corresponding factor realizations. Empirical implementation of the test to stocks in the S\&P 500 index and the five Fama–French factors, as well as the momentum factor, reveals different intraday behavior of the factor loadings: assets’ exposure to size, market and value risks vary systematically over the trading day while the three remaining factors do not exhibit statistically significant intraday variation.\end{rSubsection}
	
\end{rSection}

% Working Papers
\begin{rSection}{Work in Progress}
\begin{rSubsection}{Asymmetric Violations of the Spanning Hypothesis }{}{(with Gustavo Freire)}{}
We propose a novel decomposition of the risk premium earned by American Treasury bonds and show that the inclusion of macroeconomic data in forecasting models can significantly improve forecasting power, but only for the shorter end of the yield curve. The longer end behaves exactly like Dynamic Term Structure models in Macro-Finance would predict. We deploy several Machine Learning techniques to deal with the high-dimensionality of a comprehensive panel of macroeconomic variables used in forecasting. Our results show that standard Dynamic Term Structure models are at odds with the data for the short end of the yield curve.
\end{rSubsection}
	
\end{rSection}

\begin{rSection}{Coding Skills}
\begin{itemize}
	\item Python: experience with training, validation and deployment of ML models with \texttt{scikitlearn}; data handling with \texttt{Pandas}; scientific computing with \texttt{jax}; visualization with \texttt{matplotlib}, \texttt{plotly} and \texttt{seaborn}
	\item R: creation of reports and case studies using \texttt{R Markdown}; estimation of large-scale econometric models; experience with large panels of data and visualization with both \texttt{ggplot} and \texttt{plotly}
	\item Matlab: estimation of option pricing models; estimation of large-scale DSGE models for Macroeconomics;
	\item LaTeX and Markdown: experience with using both for technical reporting and documentation.
	\item SQL: extensive experience creating queries to access financial data on different servers
	\item Experience using UNIX systems for high-performance computations, including remote deployment, \texttt{conda environments}.
\end{itemize}

\end{rSection}

\begin{rSection}{Languages}
English: fluent; Portuguese: native; Spanish: advanced.
	
\end{rSection}

\begin{rSection}{Immigration Status}
	Nationality: Brazilian
	
	Current Status: F1-type visa until 2024, with automatic OPT extension.
\end{rSection}

\begin{rSection}{teaching experience} \itemsep -20pt

\begin{rSubsection}{Introduction to Econometrics at Northwestern Univerisity}{Fall 2021, 2022, 2023}{Core PhD course}{}
\item I assisted Prof. Viktor Todorov with a new core course for the Finance PhD degree, covering introductory topics in Statistics and Econometrics. I also hosted lectures and prepared empirical exercises.
\end{rSubsection}

\begin{rSubsection}{Investments at Northwestern Univerisity}{Spring 2023 and Fall 2023}{both for MBAs and Undergrads}{}
\item Quarter-long courses on Investments taught by Prof. Viktor Todorov both at the MBA and Undergrad level. I prepared computational exercises for students and went in depth into the more abstract concepts behind portfolio optimization.
\end{rSubsection}

\begin{rSubsection}{Derivatives at Northwestern Univerisity}{Fall 2021 and Fall 2022, Winter 2022}{both for MBAs and Undergrads}{}
\item Quarter-long courses on Derivatives taught by Prof. Viktor Todorov both at the MBA and Undergrad level.
\end{rSubsection}

\begin{rSubsection}{Data Science at FGV-EPGE for the Masters in Finance program}{March 2019 - June 2019}{MBA-Level Teaching Assistantship}{}
\item I was an assistant to Prof. Genaro Lins. MBA-level statistical reasoning both with theory and with \texttt{R} examples, including writing lecture notes in \texttt{Markdown} for classes. 
\end{rSubsection}

\begin{rSubsection}{Econometrics at FGV-EPGE for the Masters in Finance program}{January 2019 - March 2019}{MBA-Level Teaching Assistantship}{}
\item I was an assistant to Prof. Pedro Engel. I taught MBA-level statistical reasoning both with theory and with \texttt{R} examples, including writing lecture notes in \texttt{Markdown} for classes. 
\end{rSubsection}

\begin{rSubsection}{Teaching Assistant at FGV-EPGE M.Sc. and Ph.D. programs}{April 2018 - October 2018}{Statistics PhD Core Sequence}{}
\item Assisted Professors André Trindade and Marcelo Moreira for their courses on the Statistics core sequence at FGV-EPGE M.Sc. and Ph.D programs.
\end{rSubsection}


\end{rSection}


\end{document}
